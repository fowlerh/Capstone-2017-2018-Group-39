
%% bare_adv.tex
%% V1.4b
%% 2015/08/26
%% by Michael Shell
%% See: 
%% http://www.michaelshell.org/
%% for current contact information.
%%
%% This is a skeleton file demonstrating the advanced use of IEEEtran.cls
%% (requires IEEEtran.cls version 1.8b or later) with an IEEE Computer
%% Society journal paper.
%%
%% Support sites:
%% http://www.michaelshell.org/tex/ieeetran/
%% http://www.ctan.org/pkg/ieeetran
%% and
%% http://www.ieee.org/

%%*************************************************************************
%% Legal Notice:
%% This code is offered as-is without any warranty either expressed or
%% implied; without even the implied warranty of MERCHANTABILITY or
%% FITNESS FOR A PARTICULAR PURPOSE! 
%% User assumes all risk.
%% In no event shall the IEEE or any contributor to this code be liable for
%% any damages or losses, including, but not limited to, incidental,
%% consequential, or any other damages, resulting from the use or misuse
%% of any information contained here.
%%
%% All comments are the opinions of their respective authors and are not
%% necessarily endorsed by the IEEE.
%%
%% This work is distributed under the LaTeX Project Public License (LPPL)
%% ( http://www.latex-project.org/ ) version 1.3, and may be freely used,
%% distributed and modified. A copy of the LPPL, version 1.3, is included
%% in the base LaTeX documentation of all distributions of LaTeX released
%% 2003/12/01 or later.
%% Retain all contribution notices and credits.
%% ** Modified files should be clearly indicated as such, including  **
%% ** renaming them and changing author support contact information. **
%%*************************************************************************



\documentclass[10pt,peerreview,draftclsnofoot,onecolumn,final,margin=0.75in]{IEEEtran}

\usepackage{graphicx}



\begin{document}

\title{Problem Statement}

\author{Taylor Kirkpatrick \\ Group 39 \\CS 461, Fall 2017}

% The paper headers
\markboth{Problem Statement}{Taylor Kirkpatrick, Fall 2017}
\maketitle

\begin{abstract}
This document contains an overview of the state-of-the-art management and administration
software proposed as our capstone project and the expected features and functionality it 
will take. This document also highlights some preliminary estimations on viable technologies 
that can be used for this project. The document ends with an overview of stakeholders, motivations, 
and how the team may judge progress/success.

\end{abstract}
\newpage
% \IEEEpeerreviewmaketitle

\ifCLASSOPTIONcompsoc
\IEEEraisesectionheading{\section{Introduction}\label{sec:introduction}}
\else
\section{Problem Defninition}
\label{sec:Problem Defninition}
\fi
% The very first letter is a 2 line initial drop letter followed
% by the rest of the first word in caps (small caps for compsoc).
% 
% form to use if the first word consists of a single letter:
% \IEEEPARstart{A}{demo} file is ....
% 
% form to use if you need the single drop letter followed by
% normal text (unknown if ever used by the IEEE):
% \IEEEPARstart{A}{}demo file is ....
% 
% Some journals put the first two words in caps:
% \IEEEPARstart{T}{his demo} file is ....

% 
\IEEEPARstart{T}{he} project submitted for the 2017-2018 capstone year, and the one 
selected by group 39, is currently titled "Healthy Dogs." The client leading the development team is 
Dr. Chinweike Eseonu, assistant professor in the MIME department. The development team will also be 
cooperating with members of the OSU veterinary hospital during the project. Note that not much is 
currently known about the project as Dr. Eseonu is currently not in the country, and in-depth discussion 
of the project will not take place until all members of the team can meet. \\


The project is centered around the animal patients at the OSU veterinary hospital, and the management of the
OSU veterinary hospital. The hospital is currently using outdated software that is restrictive and does 
not allow hospital staff or owners of the animal patients much freedom or remote functionality. The purpose of
the project is to create a new and effective system for the hospital, allowing staff to reduce administration 
and management errors, ease suffering more quickly, and reduce stress in both pet owners and staff. The project
is envisioned to include a full management package that eases patient inflow and outflow, scheduling, internal and 
external communication, and visualizing hospital interactions. This project has not yet begun development, so the
framework, language, and other finer details of the design have yet to be selected.

\section{Proposed Solution}
As a larger team, we have yet to meet to decide how this project will be implemented. Speaking individually, I think
that we could use a Django Python framework to create a secure and unsecure website that interacts with whatever database
the hospital is currently using. Should they also want to replace the current database, I will push for a mySQL database.
A significant portion of this answer requires knowing what the hospital currently has, and what they want to keep around.
Creating an entire hospital management system may be a bit too big for a capstone project, so it is likely that the hospital
will need to keep some older things around. Another option instead of/in addition to the Django Python webpage is a server-side 
Jenkins instance running a pipeline. This will allow many users and allow tasks to be divided into small repetitive jobs that 
can be kicked off automatically.

\section{Performance Metrics}
It is vital that we meet with our client as soon as we can. This is a large project, and getting started early is paramount.
We need to attempt to meet with our client by October 13th, and aim to meet with representatives from the hospital at the same 
time or soon after. I would also like to start some development before the quarter is over, to give the team a better shot at 
recovering from the winter break without procrastinating multiple weeks at the start. Ideally, once we have a solid understanding 
of our responsibilities after making the Requirements Document, we will be at least ready to develop immediately by winter break.

As for metrics on the performance as well, our main focus will be on the hospital. They understand their own needs and the needs of pet
owners better than any of us can, so they will be our main source of feedback and criticism. If we can show them things are going to plan
as outlined, or convinced them we have a good reason for diverting from the outline, we should be able to consider ourselves "on track."

\end{document}
